% resumo em português
\setlength{\absparsep}{18pt} % ajusta o espaçamento dos parágrafos do resumo
% --- resumo em português ---
\begin{resumo}
A \emph{Internet of Things} (IoT) est\'a diretamente conectada \`a inova\c{c}\~oes tecnol\'ogicas em diversas \'areas, logo, a \'area de \emph{IoT} ir\'a crescer significantemente e isso gera a necessidade de uma aten\c{c}\~ao especial com a seguran\c{c}a nessas redes. As redes \emph{IoTs} s\~ao vulner\'aveis e suscet\'iveis \`a ataques, logo se n\~ao forem tomadas medidas a fim de garantir a seguran\c{c}a destas redes, as mesmas poder\~ao ficar indispon\'iveis e consequentemente deixar preju\'izos para empresas que utilizam sistemas IoT. As vulnerabilidades de maior impacto encontradas foram a escuta passiva, \emph{man in the middle} e o \emph{denial of service} como o \emph{flooding} e o \emph{jamming}. Este trabalho apresenta um estudo da seguran\c{c}a em redes IoT utilizando mecanismos confi\'aveis para garantir a seguran\c{c}a nestas redes e, com o objetivo de realizar isso, mecanismos como um sistema para entrega de endere\c{c}os, padr\~ao de \emph{whitelist} e criptografia s\~ao implementados. Os resultados revelam que os mecanismos aplicados s\~ao efetivamente funcionais e garantem a seguran\c{c}a b\'asica da rede, deste modo podem ser utilizados para assegurar redes IoTs. 
    
    \vspace{\onelineskip}
    \noindent
    \textbf{Palavras-chave}: IoT, Internet of Things, Security, DoS, Flooding, Jamming, Sniffing, Man in the Middle, Whitelist, Wireless Mesh Networks.
\end{resumo}

% resumo em inglês
\begin{resumo}[Abstract]
    \begin{otherlanguage*}{english}

\textit{%
Internet of Things (IoT) is directly connected to technological innovations in several areas, thus this area will grow significantly and this generates the need for special attention on security. IoT networks are vulnerable and is an easy target for attacks, so if no action is taken in order to avoid that, these networks can become unavailable and consequently leaving a disadvantage to companies who uses IoT systems. The vulnerabilities of greatest impact found were eavesdropping, man in the middle and denial of service like flooding and jamming. This work presents a study of security in an IoT network using trusted mechanisms to grant defense in these networks, in order to do that, mechanisms like a system to deliver addresses, whitelist pattern and criptography are implemented. Results reveal that mechanisms applyied are effectively worth and grant a basic security to the network. Thus can be used to secure IoTs networks.
}%

	    \vspace{\onelineskip}
	    \noindent
	    \\
	    \textbf{Keywords}: IoT, Internet of Things, Security, DoS, Flooding, Jamming, Sniffing, Man in the Middle, Whitelist, Wireless Mesh Networks.
    \end{otherlanguage*}
\end{resumo}