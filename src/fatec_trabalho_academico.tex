% Trabalho Academico FATEC-SJC v-1.1
% Criado por Marcos Hideki Inoue Junior

% -- Iniciando o documento --
\documentclass[
	12pt,		% Tamanho da fonte
	a4paper,	% Tamanho do papel
	english,	% Idioma adicional
	brazil,		% Idioma principal
	openright,	% Capitulos começam em pag impar
	oneside		% Apenas 1 página por folha
	]{abntex2}

\usepackage{lmodern}			% Usa a fonte Latin Modern
\usepackage[T1]{fontenc}		% Selecao de codigos de fonte.
\usepackage[utf8]{inputenc}		% Codificacao do documento
\usepackage{lastpage}			% Usado pela Ficha catalográfica
\usepackage{indentfirst}		% Indenta o primeiro parágrafo de cada seção.
\usepackage{color}				% Controle das cores
\usepackage{xcolor}
\usepackage{graphicx}			% Inclusão de gráficos
\usepackage{microtype} 			% para melhorias de justificação
\usepackage{lipsum}				% para geração de dummy text
\usepackage{geometry}			% para alteração no layout das páginas
\usepackage{lscape}             % colocar páginas na horizontal compativel com longtable e supertabular.
\usepackage{bookmark}
\usepackage{caption}            % adiciona \caption*{d} para criar fontes em imagens.
\usepackage{float}				% para uso no posicionamento de imagens
\usepackage{pdfpages}
\usepackage{minibox}
\usepackage{listings}
\usepackage{multirow}           % Habilita o Merge de celulas na table
\usepackage{tabularx}
\usepackage{listings} %For code in appendix
\usepackage{hyperref}
\usepackage[utf8]{inputenc}
\usepackage[T1]{fontenc}
\usepackage{inconsolata}
\usepackage{color}
\usepackage{listings}
\usepackage{color}

\usepackage{ragged2e}
 
\usepackage[pagestyles]{titlesec}

% Alterando o tamanno e fonte dos títulos
\titleformat*{\section}{\large\selectfont}
\titleformat*{\subsection}{\large\selectfont}
\titleformat*{\subsubsection}{\large\selectfont}
\titleformat{\chapter}%
  {\selectfont\bfseries\Large}{\thechapter}{4pt}{}

% Configurando trechos de codigo Java
\definecolor{verdecpp}{rgb}{0,0.5,0}
\definecolor{verdejava}{rgb}{0.25,0.5,0.35}
\definecolor{jpurple}{rgb}{0.5,0,0.35}

\lstdefinestyle{Java}
{
  language={Java},
  basicstyle=\ttfamily\small, 
  keywordstyle=\color{jpurple}\bfseries,
  stringstyle=\color{red},
  commentstyle=\color{verdejava},
  morecomment=[s][\color{blue}]{/*}{/},
  extendedchars=true, 
  showspaces=false, 
  showstringspaces=false, 
  numbers=left,
  numberstyle=\tiny,
  breaklines=true, 
  breakautoindent=true, 
  captionpos=b,
  xleftmargin=0pt,
  tabsize=4
}

% ---
% Tabela
% ---
\newcommand{\tabela}[4]{
  \begin{table}[H]
    \caption{#2 \label{#4}} 
    \centering
    #1
    \captionsetup{justification=raggedright, singlelinecheck=false}
    \fonte{#3}
  \end{table}
}

% ---
% Figura
% ---
\newcommand{\figura}[4]{
	\begin{figure}[H]
		\caption{#2 \label{#4}}
    \begin{center}
      \includegraphics[width=0.8\linewidth]{#1}
    \end{center}
    \captionsetup{justification=raggedright,singlelinecheck=false}
    \fonte{#3}
  \end{figure}
}

% ---
% Alterações no modelo original da abnTex2
% ---
% Impressão da Capa
\renewcommand{\imprimircapa}{%
  \begin{capa}%
    \center
    \ABNTEXchapterfont\large\imprimirinstituicao\\
    \vspace{5cm}
    \imprimirautor

    \vfill
    \begin{center}
    \ABNTEXchapterfont\bfseries\LARGE\imprimirtitulo
    \end{center}
    \vfill

    \large\imprimirlocal

    \large\imprimirdata

    \vspace*{1cm}
  \end{capa}
}
% ---
% Alteração na assinatura dos componentes da banca
\setlength{\ABNTEXsignwidth}{10cm}

\geometry{
 a4paper,
 bottom=2cm,
 top=3cm,
 left=3cm,
 right=2cm
}

% ---
% Configura layout para elementos textuais
\renewcommand{\textual}{%
  \pagestyle{plain}%abntheadings
  %\nouppercaseheads%
  \bookmarksetup{startatroot}%
  \pagenumbering{arabic}
}

\renewcommand{\pretextual}{%
  \pagestyle{plain}
  \pagenumbering{roman}
}

%%% -----
%%% Formato de cabeçalho/rodapé romano nos elementos pré-textuais
%%% -----

%% Novo estilo
\makepagestyle{estilo_pretextual} %%% escolha um nome
  %\makeevenhead{estilo_pretextual}{}{}{\ABNTEXfontereduzida \textbf \thepage}
  \makeoddhead{estilo_pretextual}{}{}{\ABNTEXfontereduzida \textbf \thepage}

%% Customiza comando \pretextual
\renewcommand{\pretextual}{
  \pagenumbering{roman} %%% ou \pagenumbering{Roman}
  \aliaspagestyle{chapter}{estilo_pretextual}% customizing chapter pagestyle
  \pagestyle{estilo_pretextual}
  \aliaspagestyle{cleared}{empty}
  \aliaspagestyle{part}{estilo_pretextual}
}

% ---
% Ajusta a marca \textual para que a numeração volte a ser arábica
% nos elementos textuais
\let\oldtextual\textual        % copia o comando \textual anterior para \oldtextual
\renewcommand{\textual}{%
  \pagestyle{plain}%abntheadings
  %\nouppercaseheads%
  \aliaspagestyle{chapter}{plain}
  \bookmarksetup{startatroot}%
  \pagenumbering{arabic}
}
% ---

\makeatletter
\renewcommand*{\ps@plain}{%
  \let\@mkboth\@gobbletwo
  \let\@oddhead\@empty
  \def\@oddfoot{%
    \reset@font
    \hfil
    \thepage
    % \hfil % removed for aligning to the right
  }%
  \let\@evenhead\@empty
  \let\@evenfoot\@oddfoot
}
\makeatother

% ---
% Pacotes de citações
% ---
\usepackage[brazilian, hyperpageref]{backref}	 % Paginas com as citações na bibl
\usepackage[alf]{abntex2cite}	% Citações padrão ABNT



% ---
% CONFIGURAÇÕES DE PACOTES
% ---

% ---
% Configurações do pacote backref
% Usado sem a opção hyperpageref de backref
\renewcommand{\backrefpagesname}{Citado na(s) página(s):~}
% Texto padrão antes do número das páginas
\renewcommand{\backref}{}
% Define os textos da citação
\renewcommand*{\backrefalt}[4]{
	\ifcase #1 %
		Nenhuma citação no texto.%
	\or
		Citado na página #2.%
	\else
		Citado #1 vezes nas páginas #2.%
	\fi}%
% ---

% ---
% Informações de dados para CAPA e FOLHA DE ROSTO
% ---
\titulo{Titulo do TG}

\autor{Nome Completo do Autor}
\local{São José dos Campos}
\data{\the\year}

\orientador{TituloDoOrientador. Nome do Orientador}
\coorientador{TituloDoCoorientador. Nome do coorientador}

\tipotrabalho{Trabalho de Graduação}

\newcommand{\disciplina}{Banco de Dados}

\preambulo{Trabalho de Graduação apresentado à Faculdade de Tecnologia São José dos Campos, como parte dos requisitos necessários para a obtenção do título de Tecnólogo em \disciplina.}

% Comandos para a folha de catalogacao
\newcommand{\cursoRef}{Curso de Tecnologia em Banco de Dados}
\newcommand{\instituicaoRef}{FATEC de S\~ao Jos\'e dos Campos: Professor Jessen Vidal}
\newcommand{\sobrenomeRef}{SobrenomeReferencia}
\newcommand{\nomeRef}{nomesReferencia}
\newcommand{\rgRef}{12.345.678-9}
% informações do PDF
\makeatletter
\hypersetup{
     	%pagebackref=true,
		pdftitle={\@title},
		pdfauthor={\@author},
    	pdfsubject={\imprimirpreambulo},
	    pdfcreator={LaTeX with abnTeX2},
		pdfkeywords={abnt}{latex}{abntex}{abntex2}{trabalho acadêmico},
		colorlinks=true,       		% false: boxed links; true: colored links
    	linkcolor=blue,          	% color of internal links
    	citecolor=blue,        		% color of links to bibliography
    	filecolor=magenta,      		% color of file links
		urlcolor=blue,
		bookmarksdepth=4
}
\makeatother

% ---
% Espaçamentos entre linhas e parágrafos
% ---
\setlength{\parindent}{1.3cm} % Tamanho do parágrafo

% Controle do espaçamento entre um parágrafo e outro:
\setlength{\parskip}{0.2cm}  % tente também \onelineskip

% ---
% compila o indice
% ---
\makeindex
% ---

% ----
% Início do documento
% ----
\begin{document}
\pretextual
\selectlanguage{brazil}
\frenchspacing % Retira espaço extra obsoleto entre as frases

% ---
% Configuring all citations
\citeoption{abnt-etal-list=0}
\citeoption{abnt-last-names=abnt}
\citeoption{abnt-full-initials=yes}
% ---

% ---
% Configuring listing code
% See references on internet
% ---
\lstdefinestyle{customc}{
  belowcaptionskip=1\baselineskip,
  breaklines=true,
  %frame=L,
  xleftmargin=\parindent,
  language=C,
  numbers=left,
  showstringspaces=false,
  frame=single,
  basicstyle=\footnotesize\ttfamily,
  keywordstyle=\bfseries\color{green!40!black},
  commentstyle=\itshape\color{purple!40!black},
  identifierstyle=\color{blue},
  stringstyle=\color{orange},
}

\lstset{escapechar=@,style=customc}

% ---

% ---
% Capa
% ---
\imprimircapa
% ---

% ---
% Folha de rosto
% (o * indica que haverá a ficha bibliográfica)
% ---
\thispagestyle{empty}
\imprimirfolhaderosto
% ---

% ---
% Inserir a ficha bibliografica
% ---

\begin{fichacatalografica}
    %\includepdf{./testFicha}
    \include{pre_cip}
\end{fichacatalografica}

% ---
% Folha de aprovação
% ---

% Descomentar para imprimir a folha
\include{pre_folhaAprov}

% Descomentar quando o arquivo folhaAprov.pdf estiver no diretorio raiz
%\includepdf[]{folhaAprov.pdf}
% ---

% ---
% Dedicatória
% ---
\include{pre_dedic}
% ---

% ---
% Agradecimentos
% ---
\include{pre_agrad}
% ---

% ---
% Epigrafe
% ---
\include{pre_epigrafe}
% ---

% ---
% RESUMOS
% ---
\include{pre_resumos}
% ---

% ---
% inserir lista de ilustrações
% ---
\renewcommand*\listfigurename{Lista de Figuras}
\pdfbookmark[0]{\listfigurename}{lof}
\listoffigures*
\cleardoublepage
% ---

% ---
% inserir lista de tabelas
% ---
\pdfbookmark[0]{\listtablename}{lot}
\listoftables*
\cleardoublepage
% ---

% ---
% inserir lista de abreviaturas e siglas
% ---
\include{pre_listaSiglas}
% ---

% ---
% inserir lista de símbolos
% ---
\include{pre_listaSimb}
% ---
% ---
% inserir o sumario
% ---
\pdfbookmark[0]{\contentsname}{toc}
\tableofcontents*
\cleardoublepage
% ---

% ----------------------------------------------------------
% ELEMENTOS TEXTUAIS
% ----------------------------------------------------------

\textual

\setcounter{page}{15}
% ---
% Incluindo Capitulo 1 - Introducao
% ---
% ----------------------------------------------------------
% Introdução (exemplo de capítulo sem numeração, mas presente no Sumário)
% ----------------------------------------------------------
\chapter[Introdução]{Introdução}
%\addcontentsline{toc}{chapter}{Introdução}
% ----------------------------------------------------------
\par O trabalho é motivado pela necessidade / oportunidade de...
\par Apresentar brevemente o problema para explicar a motiva\~c\~ao para o realizar.
\par Recomendável a utilização de figuras e/ou tabelas.

\section{Objetivos do trabalho}
\par Apresente claramente o seu problema e o que pretende fazer para resolver.


\section{Conteúdo do trabalho}
\par O presente trabalho está estruturado em seis Capítulos, cujo conteúdo é sucin-tamente apresentado a seguir: No Capítulo 2 é feita a fundamentação das tecnologias;15
O Capítulo 3 apresenta o desenvolvimento da solução; No Capítulo 4 são apresentadosos resultados; O Capítulo 5 apresenta as considerações finais deste trabalho a partir daanálise dos resultados obtidos





% ---

% ---
% Incluindo Capitulo 2 - Fundamentacao Teorica
\include{cap2_cont}
% ---

% ---
% Incluindo Capitulo 3 - Desenvolvimento
\include{cap3_dev}
% ---

% ---
% Incluindo Capitulo 4 - Casos de Testes
\include{cap4_cases}
% ---

% ---
% Incluindo Capitulo 5 - Conclusao
\include{cap5_concl}
% ---

% ---
% Referências bibliográficas
\bibliography{./ref/referencias}
% ---
\end{document}