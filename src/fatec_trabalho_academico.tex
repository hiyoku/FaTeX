% -- Iniciando o documento --
\documentclass[
	12pt,		% Tamanho da fonte
	a4paper,	% Tamanho do papel
	english,	% Idioma adicional
	brazil,		% Idioma principal
	openright,	% Capitulos começam em pag impar
	oneside		% Apenas 1 página por folha
	]{abntex2}

\usepackage{lmodern}			% Usa a fonte Latin Modern
\usepackage[T1]{fontenc}		% Selecao de codigos de fonte.
\usepackage[utf8]{inputenc}		% Codificacao do documento
\usepackage{lastpage}			% Usado pela Ficha catalográfica
\usepackage{indentfirst}		% Indenta o primeiro parágrafo de cada seção.
\usepackage{color}				% Controle das cores
\usepackage{xcolor}
\usepackage{graphicx}			% Inclusão de gráficos
\usepackage{microtype} 			% para melhorias de justificação
\usepackage{lipsum}				% para geração de dummy text
\usepackage{geometry}			% para alteração no layout das páginas
\usepackage{lscape}             % colocar páginas na horizontal compativel com longtable e supertabular.
\usepackage{bookmark}
\usepackage{caption}            % adiciona \caption*{d} para criar fontes em imagens.
\usepackage{float}				% para uso no posicionamento de imagens
\usepackage{pdfpages}
\usepackage{minibox}
\usepackage{listings}
\usepackage{multirow}           % Habilita o Merge de celulas na table

% ---
% Alterações no modelo original da abnTex2
% ---
% Impressão da Capa
\renewcommand{\imprimircapa}{%
  \begin{capa}%
    \center
    \ABNTEXchapterfont\large\imprimirinstituicao\\
    \vspace{5cm}
    \imprimirautor

    \vfill
    \begin{center}
    \ABNTEXchapterfont\bfseries\LARGE\imprimirtitulo
    \end{center}
    \vfill

    \large\imprimirlocal

    \large\imprimirdata

    \vspace*{1cm}
  \end{capa}
}
% ---
% Alteração na assinatura dos componentes da banca
\setlength{\ABNTEXsignwidth}{10cm}

\geometry{
 a4paper,
 bottom=2cm,
 top=3cm,
 left=3cm,
 right=2cm
}

% ---
% Configura layout para elementos textuais
\renewcommand{\textual}{%
  \pagestyle{plain}%abntheadings
  %\nouppercaseheads%
  \bookmarksetup{startatroot}%
  \pagenumbering{arabic}
}

\renewcommand{\pretextual}{%
  \pagestyle{plain}
  \pagenumbering{Roman}
}

%%% -----
%%% Formato de cabeçalho/rodapé romano nos elementos pré-textuais
%%% -----

%% Novo estilo
\makepagestyle{estilo_pretextual} %%% escolha um nome
  %\makeevenhead{estilo_pretextual}{}{}{\ABNTEXfontereduzida \textbf \thepage}
  \makeoddhead{estilo_pretextual}{}{}{\ABNTEXfontereduzida \textbf \thepage}

%% Customiza comando \pretextual
\renewcommand{\pretextual}{
  \pagenumbering{Roman} %%% ou \pagenumbering{Roman}
  \aliaspagestyle{chapter}{estilo_pretextual}% customizing chapter pagestyle
  \pagestyle{estilo_pretextual}
  \aliaspagestyle{cleared}{empty}
  \aliaspagestyle{part}{estilo_pretextual}
}

% ---
% Ajusta a marca \textual para que a numeração volte a ser arábica
% nos elementos textuais
\let\oldtextual\textual        % copia o comando \textual anterior para \oldtextual
\renewcommand{\textual}{%
  \pagestyle{plain}%abntheadings
  %\nouppercaseheads%
  \aliaspagestyle{chapter}{plain}
  \bookmarksetup{startatroot}%
  \pagenumbering{arabic}
}
% ---

\makeatletter
\renewcommand*{\ps@plain}{%
  \let\@mkboth\@gobbletwo
  \let\@oddhead\@empty
  \def\@oddfoot{%
    \reset@font
    \hfil
    \thepage
    % \hfil % removed for aligning to the right
  }%
  \let\@evenhead\@empty
  \let\@evenfoot\@oddfoot
}
\makeatother

% ---
% Pacotes de citações
% ---
\usepackage[brazilian, hyperpageref]{backref}	 % Paginas com as citações na bibl
\usepackage[alf]{abntex2cite}	% Citações padrão ABNT



% ---
% CONFIGURAÇÕES DE PACOTES
% ---

% ---
% Configurações do pacote backref
% Usado sem a opção hyperpageref de backref
\renewcommand{\backrefpagesname}{Citado na(s) página(s):~}
% Texto padrão antes do número das páginas
\renewcommand{\backref}{}
% Define os textos da citação
\renewcommand*{\backrefalt}[4]{
	\ifcase #1 %
		Nenhuma citação no texto.%
	\or
		Citado na página #2.%
	\else
		Citado #1 vezes nas páginas #2.%
	\fi}%
% ---

% ---
% Informações de dados para CAPA e FOLHA DE ROSTO
% ---
\titulo{Titulo do TG}

\autor{Nome Completo do Autor}
\local{São José dos Campos}
\data{\the\year}

\orientador{Titulo do orientador. Nome do Orientador}
\coorientador{Titulo do coorientador. Nome do coorientador}

\instituicao{%
  FACULDADE DE TECNOLOGIA DE SÃO JOSÉ DOS CAMPOS
  \par
  FATEC PROFESSOR JESSEN VIDAL}

\tipotrabalho{Trabalho de Graduação}

\newcommand{\disciplina}{Nome da Disciplina}

\preambulo{Trabalho de Graduação apresentado à Faculdade de Tecnologia São José dos Campos, como parte dos requisitos necessários para a obtenção do título de Tecnólogo em \disciplina.}

% Comandos para a folha de catalogacao
\newcommand{\cursoRef}{Curso de Tecnologia em \disciplina}
\newcommand{\instituicaoRef}{FATEC de S\~ao Jos\'e dos Campos: Professor Jessen Vidal}
\newcommand{\sobrenomeRef}{Sobrenome referência}
\newcommand{\nomeRef}{Nomes referência}
\newcommand{\rgRef}{00.000.000-0} % Número do RG

% informações do PDF
\makeatletter
\hypersetup{
     	%pagebackref=true,
		pdftitle={\@title},
		pdfauthor={\@author},
    	pdfsubject={\imprimirpreambulo},
	    pdfcreator={LaTeX with abnTeX2},
		pdfkeywords={abnt}{latex}{abntex}{abntex2}{trabalho acadêmico},
		colorlinks=true,       		% false: boxed links; true: colored links
    	linkcolor=blue,          	% color of internal links
    	citecolor=blue,        		% color of links to bibliography
    	filecolor=magenta,      		% color of file links
		urlcolor=blue,
		bookmarksdepth=4
}
\makeatother

% ---
% Espaçamentos entre linhas e parágrafos
% ---
\setlength{\parindent}{1.3cm} % Tamanho do parágrafo

% Controle do espaçamento entre um parágrafo e outro:
\setlength{\parskip}{0.2cm}  % tente também \onelineskip

% ---
% compila o indice
% ---
\makeindex
% ---

% ----
% Início do documento
% ----
\begin{document}
\pretextual
\selectlanguage{brazil}
\frenchspacing % Retira espaço extra obsoleto entre as frases

% ---
% Configuring all citations
\citeoption{abnt-etal-list=0}
\citeoption{abnt-last-names=abnt}
\citeoption{abnt-full-initials=yes}
% ---

% ---
% Configuring listing code
% See references on internet
% ---
\lstdefinestyle{customc}{
  belowcaptionskip=1\baselineskip,
  breaklines=true,
  %frame=L,
  xleftmargin=\parindent,
  language=C,
  numbers=left,
  showstringspaces=false,
  frame=single,
  basicstyle=\footnotesize\ttfamily,
  keywordstyle=\bfseries\color{green!40!black},
  commentstyle=\itshape\color{purple!40!black},
  identifierstyle=\color{blue},
  stringstyle=\color{orange},
}

\lstset{escapechar=@,style=customc}

% ---

% ---
% Capa
% ---
\imprimircapa
% ---

% ---
% Folha de rosto
% (o * indica que haverá a ficha bibliográfica)
% ---
\thispagestyle{empty}
\imprimirfolhaderosto
% ---

% ---
% Inserir a ficha bibliografica
% ---

\begin{fichacatalografica}
    %\includepdf{./testFicha}
    \include{pre_cip}
\end{fichacatalografica}

% ---
% Folha de aprovação
% ---

% Descomentar para imprimir a folha
\include{pre_folha_aprovacao}

% Descomentar quando o arquivo folhaAprov.pdf estiver no diretorio raiz
%\includepdf[]{folhaAprov.pdf}
% ---

% ---
% Dedicatória
% ---
\include{pre_dedicatoria}
% ---

% ---
% Agradecimentos
% ---
\include{pre_agradecimentos}
% ---

% ---
% Epigrafe
% ---
\begin{epigrafe}
    \vspace*{\fill}
    \begin{flushright}
% Little easter egg. 
%        \textit{''Eu acredito que os HotDogs simbolizam uma filosofia muito profunda, \\
%                na qual todos nós poderíamos ter a oportunidade de observá-la e aprendê-la. \\
%                O pão pode significar tudo que dá a nossa base para a nossa vida. \\
%                E a salsicha pode simbolizar nós mesmos, ou um simples conceito ou coisa que surge de uma epifania.''\\
%                (Rodrigo Takeshi, 26 de Outubro de 2016)}

        \textit{“Epigrafe, citar alguma frase de outra pessoa que tenha rela\~c\~ao com o TG”\\
        (Marcos Hideki)}
    \end{flushright}
\end{epigrafe}
% ---

% ---
% RESUMOS
% ---
% resumo em português
\setlength{\absparsep}{18pt} % ajusta o espaçamento dos parágrafos do resumo
% --- resumo em português ---
\begin{resumo}
A \emph{Internet of Things} (IoT) est\'a diretamente conectada \`a inova\c{c}\~oes tecnol\'ogicas em diversas \'areas, logo, a \'area de \emph{IoT} ir\'a crescer significantemente e isso gera a necessidade de uma aten\c{c}\~ao especial com a seguran\c{c}a nessas redes. As redes \emph{IoTs} s\~ao vulner\'aveis e suscet\'iveis \`a ataques, logo se n\~ao forem tomadas medidas a fim de garantir a seguran\c{c}a destas redes, as mesmas poder\~ao ficar indispon\'iveis e consequentemente deixar preju\'izos para empresas que utilizam sistemas IoT. As vulnerabilidades de maior impacto encontradas foram a escuta passiva, \emph{man in the middle} e o \emph{denial of service} como o \emph{flooding} e o \emph{jamming}. Este trabalho apresenta um estudo da seguran\c{c}a em redes IoT utilizando mecanismos confi\'aveis para garantir a seguran\c{c}a nestas redes e, com o objetivo de realizar isso, mecanismos como um sistema para entrega de endere\c{c}os, padr\~ao de \emph{whitelist} e criptografia s\~ao implementados. Os resultados revelam que os mecanismos aplicados s\~ao efetivamente funcionais e garantem a seguran\c{c}a b\'asica da rede, deste modo podem ser utilizados para assegurar redes IoTs. 
    
    \vspace{\onelineskip}
    \noindent
    \textbf{Palavras-chave}: IoT, Internet of Things, Security, DoS, Flooding, Jamming, Sniffing, Man in the Middle, Whitelist, Wireless Mesh Networks.
\end{resumo}

% resumo em inglês
\begin{resumo}[Abstract]
    \begin{otherlanguage*}{english}

\textit{%
Internet of Things (IoT) is directly connected to technological innovations in several areas, thus this area will grow significantly and this generates the need for special attention on security. IoT networks are vulnerable and is an easy target for attacks, so if no action is taken in order to avoid that, these networks can become unavailable and consequently leaving a disadvantage to companies who uses IoT systems. The vulnerabilities of greatest impact found were eavesdropping, man in the middle and denial of service like flooding and jamming. This work presents a study of security in an IoT network using trusted mechanisms to grant defense in these networks, in order to do that, mechanisms like a system to deliver addresses, whitelist pattern and criptography are implemented. Results reveal that mechanisms applyied are effectively worth and grant a basic security to the network. Thus can be used to secure IoTs networks.
}%

	    \vspace{\onelineskip}
	    \noindent
	    \\
	    \textbf{Keywords}: IoT, Internet of Things, Security, DoS, Flooding, Jamming, Sniffing, Man in the Middle, Whitelist, Wireless Mesh Networks.
    \end{otherlanguage*}
\end{resumo}
% ---

% ---
% inserir lista de ilustrações
% ---
\pdfbookmark[0]{\listfigurename}{lof}
\listoffigures*
\cleardoublepage
% ---

% ---
% inserir lista de tabelas
% ---
\pdfbookmark[0]{\listtablename}{lot}
\listoftables*
\cleardoublepage
% ---

% ---
% inserir lista de abreviaturas e siglas
% ---
\include{pre_lista_siglas}
% ---

% ---
% inserir lista de símbolos
% ---
\include{pre_lista_simbolos}
% ---
% ---
% inserir o sumario
% ---
\pdfbookmark[0]{\contentsname}{toc}
\tableofcontents*
\cleardoublepage
% ---

% ----------------------------------------------------------
% ELEMENTOS TEXTUAIS
% ----------------------------------------------------------

\textual

\setcounter{page}{15}
% ---
% Incluindo Capitulo 1 - Introducao
% ---
\include{capitulos/1_introducao}

% ---

% ---
% Incluindo Capitulo 2 - Fundamentacao Teorica
\include{capitulos/2_fundamentacao}
% ---

% ---
% Incluindo Capitulo 3 - Desenvolvimento
	\newpage
\chapter{Desenvolvimento}
\label{ch:desenvolvimento}

\section{Desenvolvimento dessa section}
\par Este capítulo apresenta o processo de desenvolvimento da solução proposta. Seu conteúdo pode variar, dependendo da metodologia adotada.

% ---

% ---
% Incluindo Capitulo 4 - Casos de Testes
\newpage
\chapter{Casos de Testes}
Neste capitulo serão apresentados os testes que foram implementados com a solução e o conteúdo apresentado.

% ---

% ---
% Incluindo Capitulo 5 - Conclusao
    \newpage
\chapter{Conclus\~ao}
Conclusao para o trabalho, mostra como a solu\~c\~ao proposta cumpre com o que foi apresentado anteriormente.
% ---

% ---
% Referências bibliográficas
\bibliography{./ref/referencias}
% ---
\end{document}
