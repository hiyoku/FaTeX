% ----------------------------------------------------------
% Introdução (exemplo de capítulo sem numeração, mas presente no Sumário)
% ----------------------------------------------------------
\chapter[Introdução]{Introdução}
%\addcontentsline{toc}{chapter}{Introdução}
% ----------------------------------------------------------
\par A maior parte da vida moderna depende dos computadores, das redes de computadores e atualmente a mais notável interação do ser humano é o deslizar do dedo em uma \emph{smartphone}. Essas tecnologias tendem a se integrar ainda mais com a nossa realidade com o avan\c{c}o tecnol\'ogico dos microcomputadores e microcontroladores, tornando assim o termo \emph{Internet of Things} (\emph{IoT}) cada vez mais conhecido pela sua peculiaridade de objetos se conectando \`a internet.

\par \emph{IoT} não é uma tecnologia nova, entretanto tem ganho maior espa\c{c}o com a necessidade das pessoas de receberem informa\c{c}\~oes sobre diversas coisas do seu dia a dia em tempo real, logo o termo significa nada mais do que objetos que realizam a\c{c}\~oes ou geram informa\c{c}\~oes conectados \`a internet. A principal ideia de \emph{IoT} vem da presença de coisas ou objetos de nosso dia a dia que são capazes de interagir e cooperar entre eles a fim de alcançar um objetivo em comum. A principal força da \emph{Internet of Things} vem da ideia na qual a mesma terá um grande impacto em diversos aspectos de nosso dia a dia e no comportamento de seus usuários, já que diversos cenários serão afetados por ela, como por exemplo o recebimento de informa\c{c}\~oes em tempo real de hidr\^ometros, monitoramento de vazamentos e entre diversos outros, com todas essas informa\c{c}\~oes dispon\'iveis na nuvem \'e poss\'ivel acess\'a-las de diversas maneiras, seja ela por um computador ou um \emph{smartphone}. Há diversas definições parecidas de \emph{IoT}, e atualmente muitas pessoas tem dificuldades para entender realmente o que significa seu conceito, suas ideias básicas e suas implicações sociais, econômicas e técnicas \cite{iot2005itu}.

\par Segundo \citeonline{Atzori2010a}, o que gera confus\~ao com o termo \emph{IoT} é o termo em sí, que une \emph{Internet} e \emph{Things}, dando uma ideia de conectividade que une coisas ou objetos. Expressa uma ideia genérica de objetos onde todos eles est\~ao unidos por um meio em comum, todavia há pequenas diferenças nos pontos de vista dos componentes da \emph{IoT}. O cenário como pode ser visto na \autoref{fig:iot} é dividido em 3 visões:
\begin{itemize}
    \item Visão orientada aos objetos, que são objetos do nosso dia a dia, como sensores, cart\~oes de \emph{Radio-Frequency Identification (RFID)} e entre outros.
    \item Visão orientada a Internet, trata da visão das redes ou da internet em si no contexto de \emph{IoT}, pode ser chamada de \emph{web} das coisas.
    \item Visão orientada a Semântica, visa os conceitos tecnol\'ogicos e te\'oricos em si.
\end{itemize}
Todas essas vis\~oes tornam as aplicações de \emph{IoT} muito abrangentes possibilitando uma variedade enorme de sistemas que poderão ser desenvolvidos. As principais atuações em nossas vidas estão em casa \emph{(Smart Home)} e em cidades \emph{(Smart Cities)}. São diversos domínios nas aplicações dessa tecnologia, e elas são agrupadas em: Transporte, Saúde, Ambientes Inteligentes (Escritórios, Casas) e Pessoal ou social.

\begin{figure}[ht]
	\caption{Cen\'ario da \emph{Internet of Things} com a diverg\^encia de vis\~oes}
	\centering
		\includegraphics[width=\textwidth,height=\textheight, keepaspectratio]{figuras/iot01}
    \label{fig:iot}	
	\fonte{Adaptada pelo autor, de \citeonline{Atzori2010a}}
\end{figure}

\par A criação da \emph{Internet of Things} de acordo com \citeonline{iot2005itu} está diretamente relacionada as inovações tecnológicas em diversos campos. As principais tecnologias vinculadas com \emph{IoT} são as de identificação, sensores, tecnologias inteligentes e nanotecnologia.
As tecnologias de identificação vem das \emph{tags} de \emph{RFID}, onde é possível gerar uma identificação por radiofrequência.
As tecnologias de sensores são dispositivos eletrônicos que conseguem sensorear o meio e responder à certos estímulos, seja ele mecânico, térmico, eletroestático, eletromagnético, radiação, químico, biológico e entre outros.
As tecnologias inteligentes est\~ao muito presentes no mercado, sejam eles \emph{smartphones}, leitores biométricos ou até mesmo carros inteligentes.

\par O segmento de \emph{IoT} de acordo com \citeonline{Atzori2010a} está em alta e muitas empresas estão competindo entre si para alcançar a liderança e crescer no mercado. Esse segmento é algo que pode ser utilizado desde equipamentos domésticos até em hospitais sendo composta de três componentes principais: as coisas, as redes de comunicação que as conectam e os sistemas de computação que usam os dados que fluem das coisas. A maior parte desses dispositivos comunicam-se através de um tipo de rede sem fio denominada \emph{Wireless Mesh Networks} (WMNs) ou redes \emph{mesh} sem fio.

\par As \emph{WMNs} não dependem diretamente de um \emph{Access Point} (AP) que envia e recebe as informa\c{c}\~oes para se manter na rede similar a topologia estrela. Essas redes suportam saltos e utilizam a topologia malha. Este tipo de rede é descentralizada, relativamente barata, muito confiável e resiliente, desde que cada n\'o apenas transmita para outro n\'o de sua rede. Esses n\'os agem como repetidores para transmitir as informações até o seu destino, muito útil para longas distâncias \cite{siddiqui2007}.

\par Atualmente com a tecnologia avançada, as empresas e organizações estão mais dependentes de seus computadores e dispositivos, logo atrai a aten\c{c}\~ao de pessoas mal intencionadas. As ameaças de criminosos e terroristas para os sistemas da informação estão aumentando e empresas de diversos portes sentem a necessidade de proteger seus dados e seus sistemas. Esta necessidade é vital para o negócio, pois grande valor de uma empresa pode estar em seus dados ou toda uma operação pode depender do bom funcionamento de seus equipamentos. As ameaças cibernéticas estão sempre em constante avanço, buscando serem efetivos e cada vez mais sofisticados, em contra partida s\~ao utilizadas tecnologias atualizadas, pol\'iticas e pr\'aticas com o objetivo de evitar possíveis ameaças a integridade e disponibilidade das informa\c{c}\~oes \cite{Malgeri2009a}.

\par Apenas 40\% das empresas que utilizam \emph{IoT} já implementaram alguma medida de segurança, contudo a pesquisa revela que 92\% dos usuários de \emph{IoT} estão preocupados com a segurança. Isso abrirá oportunidades para abordagens de segurança mais robustas, não visando apenas empresas, mas também dispositivos voltados para o ambiente doméstico \cite{KPMG2015a}.

\par H\'a casos como o da \emph{botnet} Mirai, na qual diversos dispositivos \emph{IoT} foram comprometidos fazendo-o com que realizassem um grande ataque \emph{DDoS (Distributed Denial Of Service)} contra a empresa Akamai Technologies que trabalha com \emph{cloud computing}. De acordo com a empresa, foi o maior \emph{DDoS} registrado por eles, sendo ele de 667 Gigabits de tr\'afico por segundo \cite{miraibotnet}. N\~ao foi citado muito dos dispositivos \emph{IoT} utilizados no ataque, entretanto, \'e subjetivamente compreendido que os dispositivos n\~ao tinham a seguran\c{c}a devida implementada, ou n\~ao foi efetivamente testada, ou a seguran\c{c}a dos dispositivos foram burladas.

\section{Problema}
\par Vulnerabilidades e falhas de segurança em redes de \emph{IoT}.
%\par Este trabalho tem como objetivo analisar os componentes de uma rede de \emph{IoT} e levantar as vulnerabilidades a fim de realizar medidas para tornar uma rede mais segura para o provável uso em \emph{Smart Homes} e até em \emph{Smart Cities}.
%\subsection{Objetivos Espec\'ificos} Opcional
\section{Objetivo Geral}
\par Este trabalho tem como objetivo analisar e levantar as possíveis ameaças e vulnerabilidades de redes de \emph{IoT} que utilizam a topologia \emph{mesh} a fim de buscar e tentar aplicar medidas e tecnologias para evitar possíveis falhas de segurança, de privacidade e perda de informações, para tornar o uso de redes de \emph{IoT} mais seguros.

\section{Objetivo Espec\'ifico}
\par Levantar poss\'iveis amea\c{c}as e vulnerabilidades no protocolo \emph{Lightweight Mesh} que tem como base IEEE 802.15.4, buscar medidas e tecnologias para levar poss\'iveis maneiras de mitiga\c{c}\~ao das mesmas. 
